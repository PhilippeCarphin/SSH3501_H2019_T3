\documentclass[11pt]{article}


\usepackage{setspace}
\newcommand{\philnewpage}{}
% \newcommand{\philnewpage}{\newpage}
\usepackage{geometry}
\usepackage{apacite}
\usepackage{setspace}
\linespread{1.5}
\geometry{letterpaper,tmargin=1in,bmargin=1in,lmargin=1.25in,rmargin=1.25in}

\usepackage[utf8]{inputenc}
\usepackage[T1]{fontenc}
\usepackage[frenchb]{babel}

\usepackage[final]{pdfpages}
\begin{document}

\includepdf[pages=1]{page_titre.pdf}
\tableofcontents
\setcounter{page}{1}
\newpage

\section{Introduction}

  En début 2018, le journal \emph{The Guardian}, qui reçevait des informations
  privilégiées de Christopher Wylie, un ex-employé de la firme Cambridge Analytica

  En début 2018, le scandal concernant la firme Cambridge Analytica et Facebook
  tombe dans l'oeil publique avec la publication par \emph{The Guardian} de
  l'article \ref{}.

  % Cet article, et d'autres montre que Cambridge Analytica a obtenu de façon
  % illicite des données, et dénoncent aussi les usages faits de ces données

  % Je vais me concentrer sur les usages faits des données

  % Décrire en grand détail l'extent des choses qu'ils font

  % Créent des profils psychologiques

  % Basicall reproduce the article where chris wylie complains about what CA does

  % Et ma position spontanée c'est que de telles choses sont pas éthiques.


  % Christopher Wylie trouve ça pas nice.  Il voit aussi le potentiel ...


  % Qui d'autre trouve ça pas nice, le public pas trop, mais rappelons que je ne
  % me concentre pas sur la façon dont les données sont acquises.

  % L'autre qui a témoigné à CPAN a rappelé de pas capoter et que c'est pas du
  % mind control quand même.


  % Qui pense que c'est chill


  % Nous allons étudier cette ce sujet en utilisant plusieurs théories éthiques.
  % Nous allons commencer par donner un survol des systèmes normatifs qui sont
  % en jeux pour ensuite faire une analyse selon le conséquentialisme et
  % l'éthique du devoir.

  % Nous allons terminer par une analyse au niveau des fondements de la morale
  % et une analyse au niveau de la motivation morale

\philnewpage
\section{Systèmes normatifs}

Les systèmes normatifs en général regroupent tous les énoncés qui prescrivent.
Ces énoncés sont typiquement regroupés en catégories.

Nous avons les lois qui sont des énoncés qui disent comment les citoyens doivent
agir pour vivre dans la société.  Les meurs sont des énoncés qui disent comment
une personne doit agir pour être acceptée dans des cercles sociaux, et
finalement la religion (qui dans nos temps modernes peut être remplacée par une
idée floue de ce qui est vu comme sacré par le groupe).

Le système normatif le plus évident dans ce scandal est le système judiciaire.
En effet des lois ont étées brisées.

% Raconter toutes les lois qui ont étées brisées

Par contre, ces lois concernent l'acquisition de données, qui est un aspect
périphérique à notre question centrale.

En ce moment, les lois concernent principalement l'échange de données.  Il est
même raisonnable de croire qu'il n'y aurait même pas eu de scandal si les
données dont il est question avaient étées acquises de façon légitime.

Pourtant si on regarde le discours de Christopher Wylie, on voit qu'une très
bonne partie de ses propos sont indépendants de comment les données sont
acquises.

% Bunch of quotes from Chris Wylie


\philnewpage
\section{Analyse selon l'éthique conséquentialiste}

L'éthique conséquentialiste étudie les actions en fonction de leur conséquences
avec le bonheur et le plaisir comme mesure principale.

On considère que les mécanismes de bonheur et de plaisir ont une origine
fonctionnelle, celle de nous orienter vers notre but. En ce sens, si les
plaisirs que nous ressentons existent, comme le plaisir des relations sexuelles,
pour nous inciter à faire quelque chose dans l'intérêt de l'espèce, nous pouvons
voir les plaisirs et le bonheur comme des outils prescriptifs.

Le conséquentialisme voit le plaisir et le bonheur comme la mesure commune pour
juger les actions. La théorie conséquentialiste ajoute quelques raffinements
pour permettre son application aux sociétés. Par exemple, on considère qu'une
action est bonne si elle maximise le bonheur du plus grand nombre. Une analyse
conséquentialiste sera donc un bilan comptable du bonheur.


\philnewpage
\section{Analyse selon l'éthique du devoir}

Nous contrastons l'analyse conséquentialiste avec l'éthique du devoir de Kant,
pour qui les conséquences n'ont pas d'importance.

Voyant la religion comme la chose qui pousse les gens à faire le bien, et voyant
que son influence déclinait, Kant s'est interrogé sur le fondement le plus
profond du bien.

Il a tenté d'accéder à l'idée du bien par raison pure.  En postulant que
l'humain a une valeur intrinsèque, Kant en arrive aux principes de l'impératif
catégoriques: cohérence, égale dignité, autonomie.

Commençons par le principe de cohérence, celui-ci dit qu'une action doit être
universalisable.  Dans notre cas, pour que participer à des campagnes de
manipulations d'opinion public soit une action moralement acceptable, il
faudrait accepter qu'on soit la cible de telles campagnes.  Or puisque ni moi,
ni Christopher Wylie ne voulons nous faire présenter du contenu qui a été choisi
pour servir les intérêts de quelqu'un d'autre au détriment des mien, je ne veux
pas être la cible de telles campagnes.

Ensuite, le principe d'égale dignité traduit la valeur intrinsèque de l'humain.
Celui-ci dit de traiter les autres comme une fin en eux-mêmes et non un moyen.
Ce principe est violé de façon évidente.  Cambridge Analytica font des campagnes
pour quiconque les engage.  Si ce n'était pas le cas, on pourraît dire, pour
l'exemple de l'interférence avec les élections, que si CA croit légitimement que d'avoir
Trump comme président est pour le bien de tous, alors ils respectent le principe
d'égale dignité.

Ceci nous amène au principe d'autonomie.  Ce principe parle de la liberté des
humains.  Puisque Kant se concentre sur les intentions, et non les conséquences,
\philnewpage
\section{Concept 4}

Nous quittons le monde normatif et prescriptif pour entrer dans le monde
descriptif.  Nous commençons par cette section qui est une analyse selon le
modèle \emph{communauté, autonomie, divinité} (CAD).  Ensuite la section suivante analysera
les motivations morales des différents acteurs.

Le modèle CAD est une façon d'analyser une situation morale selon trois
facettes.  L'aspect \emph{communauté} explore la relation entre l'individu et le
groupe, ce son rôle face à celle-ci et les impacts de nos actions sur celle-ci.
L'aspect \emph{autonomie} explore la relation entre l'individu et d'autres individus
(incluant lui-même).
Finalement, l'aspect \emph{divinité} explore la relation entre l'individu et le
sacré.  Notons que le mot sacré dans ce cas réfère à ce qui est plus grand que
soi comme la nature elle-même.

Commençons par analyser la composante d'autonomie.  En tant qu'ingénieur, je
recherche un emploi payant et en tant que passioné de mon domaine, je veux faire
des accomplissements technologiques.  En tant qu'humain, je souhaite aussi être
reconnu pour mes accomplissements.

Ainsi, en travaillant dans grande firme de marketing, je ferais un bon salaire,
et si, comme Christopher Wylie, je peux me retrouver comme leadeur d'une équipe,
j'y trouverai un sens d'accomplissement personnel dans le respect que ma
position me donne.

De plus, je peux avoir des innovations à mon nom et acquérir une renommée
puisque j'aurais participé à la création de technologies révolutionnaires.

Maintenant, regardons la relation de l'individu à la communauté.  Nous
distinguerons deux communautés: l'entreprise, et le publique.  En tant
qu'ingénieur, j'aurai des rôles dans ces deux communautés.

Considérons le rôle de l'ingénieur face à l'entreprise.  Ce rôle est donné en
partie implicitement et explicitement par les termes de l'embauche, et aussi par
le code de déontologie de l'ingénieur.  Mon rôle en tant qu'ingénieur serait
d'accomplir les tâches de mon emploi au mieux de mes capacités.  Mon rôle me
demande aussi de respecter la confidentialité des informations de l'entreprise.

Ensuite, le rôle de l'ingénieur face au publique et à la société en général.
L'ingénieur doit veiller au bien du public et faire ce qu'il croit être dans
l'intérêt de celui-ci.  

En donnant des informations au journal \emph{ The Guardian },
Chris Wylie a dévié des attentes de son rôle d'ingénieur-employé pour satisfaire les
attentes contradictoires de son rôle envers le public.  En effet, un des rôles
de l'ingénieur est de protéger le public.

Finalment en ce qui concerne l'aspect \emph{divinité}, disons simplement qu'on
peut attribuer un caractère sacré à la liberté des individus de former leurs
propres opinions.  Ainsi, un devoir fondamental d'être humain (et non un devoir
d'ingénieur ou un devoir social conféré par les attentes de la société) pourrait
nous pousser à opposer toute forme de manipulation clandestine d'opinion.

\philnewpage
\section{Concept 5}

Nous terminons avec une analyse au niveau de la motivation morale.  Nous allons
analyser la situation de deux points de vue: l'éthique de responsabilité et
l'éthique de conviction.

L'éthique de la responsabilité est de nature téléologique.  Elle est centrée sur
les buts de nos actions.

L'éthique de conviction, en contraste, est centrée sur les intentions et se
fonde sur les valeurs axiologiques plutôt que téléologiques.

Commençons par l'éthique de la responsabilité.  Quels sont les buts.  Ét bien je
veux avoir un emploi pour vivre comfortablement.  Et les gens pour qui je
travaillent veulent que leur entreprise jouisse de succès.  Et les clients de ma
firme veulent gagner leur élection ou vendre leur produit.

De l'autre côté, quels sont les valeurs intrinsèques et les intentions derière
les actes que l'on voit.  Nous pouvons citer le sentiment d'accomplissement
qu'un ingénieur peut ressentir lorsqu'il gagne le respect de ses pairs et la
fierté pour les technologies qu'il développe.  Nous pouvons aussi accorder une
valeur intrinsèque à la liberté de faire nos propres choix.

Les clients de l'entreprise et l'entreprise elle-même veulent atteindre
l'excellence dans leur industrie (ou peuvent le vouloir) simplement pour la
valeur intrinsèque qu'apporte la notoriété d'être un chef de file dans une
industrie.  Ils ont donc une motivation axiologique de poursuivre le succès.

Pour conclure cette section, disons que la recherche de l'excellence motive
l'ingénieur, l'entreprise (CA) et les clients de celle-ci.  Il s'agit d'une
motivation axiologique.

L'ingénieur a aussi une devoir envers le public qui peut être motivé de façon
axiologique par un sens de loyauté et d'honnêteté, ou de façon téléologique par
un désir d'éviter des sanctions pour avoir participé à des actes douteux.

Finalement le public a ses propres désirs et motivations.  Notamment, il désire
assurer sa propre protection.  Nous pouvons voir les comparutions de Christopher
Wylie devant le Scénat Américain et autres corps législatifs comme des
manifestations du désir du public d'être informé de ces problématiques.

On peut y voir une motivation téléologique: dans le but d'être informé des
enjeux qui pourraient avoir des impacts sur les vies privées des individus ou
faciliter des manipulations électorales, le Scénat fait venir Chris Wylie et
Mark Zuckerberg pour leur poser des questions.

Mais en arrière se cache des motivations axiologiques: nous valorisons de façon
intrinsèque (du moins je crois que c'est le cas pour la majorité) la vie privée
et la liberté de décision.

Voilà donc un résumé très bref des principales motivations en jeu dans notre
situation.  Nous avons les marketeurs et leurs clients qui recherchent pour des
raisons téléologiques le succès matériel, et recherchent aussi l'excellence et
l'accomplissement pour leur valeur intrincèque.

L'ingénieur a les mêmes motivations, mais en plus voit une valeur axiologique
dans la protection du public

Le public a ses propres motivations aussi, il veut préserver la vie privée des
individus et leur libre arbitre.

\philnewpage
\section{Conclusion}
\newpage
\section*{Références}
\setlength{\parindent}{0in}
\singlespacing

    
\singlespacing
Wylie, C. (2018). \emph{Cambridge Analytica and Data Privacy} (Chuck
Grassley intervieweur) [Fichier vidéo].  Tiré de
\texttt{https:\allowbreak/\allowbreak /www.c-span.org\allowbreak
/video\allowbreak /?445621-1\allowbreak /cambridge-\allowbreak
analytica-\allowbreak whistleblower-\allowbreak christopher-\allowbreak
wylie-\allowbreak testifies-\allowbreak data-\\privacy}

\singlespacing
Matz, S.C., Kosinski, M., Nave, G., \&Stillwell, D.J., {Psychological targeting as an effective approach to digital mass persuasion}. \emph{Proceedings of the National Academy of Science}, 114(48), 12714-12719.

\singlespacing
Therrien, D. \emph{Comparution devant le Comité permanent de l’accès à
l’information, de la protection des renseignements personnels et de l’éthique
(ETHI) au sujet de l’atteinte à la sécurité des renseignements personnels
associée à Cambridge Analytica et à Facebook}, Ottawa, Ontario.  Tiré de
\texttt{https://\allowbreak www.\allowbreak ourcommons.\allowbreak ca/\allowbreak
DocumentViewer/\allowbreak en/42-1/\allowbreak ETHI/\allowbreak report-17/}

\singlespacing
Kant, E. (2004). \emph{Fondements de la méthphysique des moeurs}. Paris, France.
J Vrin

\end{document}
